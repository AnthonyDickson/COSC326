\documentclass[12pt]{article} 
\begin{document}
\begin{LARGE}
\textbf{
\begin{center}
Etude 6 - Numbers
\end{center}}
\end{LARGE}
\section{Harmonic Numbers}
\paragraph{Definition}
	The harmonic numbers are defined as the initial partial sums of the series:
	{\large $$\frac{1}{1} + \frac{1}{2} + \frac{1}{3} + \frac{1}{4} + ....$$}\\
	The above can also be represented as follows:
	{\large $$H(n)\ = \ \frac{1}{1} \ + \ \frac{1}{2} \ + \ \frac{1}{3} \ + \ .... \ +  \ \frac{1}{n-1} \ + \ \frac{1}{n}$$}
	
\paragraph{Hypothesis}
	We expect that it makes no difference in what order we add terms together.

\paragraph{Method}
	Using the above formula, we created a java application which calculates $H(n)$ using both single and double precision floating point data types. Our java application calculates $H(n)$ in both increasing order ($\frac{1}{1} + \frac{1}{2} +  .... + \frac{1}{n}$) and decreasing order ($\frac{1}{n} + \frac{1}{n - 1} +  .... + \frac{1}{1}$). The results for calculating $H(10000)$ are shown below in figure 1.1.
	
\begin{figure}[h]
	\textbf{Figure 1.1: Results of calculating $H(10000)$.}\\
	\textbf{Increasing order}\\
	Single Precision: $9.787613$\\
	Double Precision: $9.787606036044348$\\
	\textbf{Decreasing order}\\
	Single Precision: $9.787604$\\
	Double Precision: $9.787604$	
\end{figure}

\paragraph{Observations}
As we can see in the results above, the results for $H(10000)$ calculated in increasing order are different when using single and double precision numbers. Most notably, the single precision number has about half the number of digits of the double precision number. However, the single and double precision results are the same when  $H(10000)$ was calculated in decreasing order. Furthermore, the single precision results and the double precision results are different based on the order that the terms were added together. These obversations can be seen in $H(n)$ for values on either side of $10000$ as shown in below in figure 1.2 and figure 1.3.

However, the single and double precision results for decreasing order in figure 1.3 show a small difference ($0.000001$). So perhaps $H(n)$ calculated in decreasing order is infact different for single and double precision numbers (albeit very slightly and perhaps not significantly).


\begin{figure}[h]
\textbf{Figure 1.2: Results of calculating $H(1000)$.}\\
\textbf{Increasing order}\\
Single Precision: $7.48547843$\\
Double Precision: $7.485470860550343$\\
\textbf{Decreasing order}\\
Single Precision: $7.4854717$\\
Double Precision: $7.4854717$
\end{figure}
	
\begin{figure}[h]
	\textbf{Figure 1.3: Results of calculating $H(100000)$.}\\
	\textbf{Increasing order}\\
	Single Precision: $12.090851$\\
	Double Precision: $12.090146129863335$\\
	\textbf{Decreasing order}\\
	Single Precision: $12.090153$\\
	Double Precision: $12.090152$	
\end{figure}	

\paragraph{}


\newpage
\section{Standard Deviation}
\paragraph{}
As per given in the assignment, Standard deviation of a series containing numbers from $1, \ 2, \ 3, \ ,4 \ ,5 \ ,6 \ ,7 \ ,8 \ ,9 \ ..... \ (n-1), \ n$ is calculated for single precision and double precision data types using Java Code(SDeviation class). It has implemented two methods to find Standard Deviation using two different way given in Assignment. Output of the code is as per following for $n=10000$.

\begin{figure}[h]
	Output:\\
	Method-1 Output \\
	Single Precision: $2886.752$ \\
	Double Precision: $2886.751331514372$ \\
	Method-1 Output with added Fixed Value [ $2.0$ ] \\
	Single Precision: $2886.7524$ \\
	Double Precision: $2886.7520243346153$
\end{figure}

\begin{figure}[h]
	Method-2 Output \\
	Single Precision: $2886.7517$  \\
	Double Precision: $2886.751331514372$ \\
	Method-2 Output with added Fixed Value [ $2.0$ ] \\
	Single Precision: $2886.7515$ \\
	Double Precision: $2886.751331514372$
\end{figure}

\paragraph{} Based on following output, for standard deviation precision matters. For small number in the array standard deviation doesn't varie much. But it varies with difference of 1.00 or more for single and double precision data types. This code also includes scenario where a Fixed value is added to every member of the series and calulated standard deviation. It doesnt vary in case of addition of same fixed value to every member in the series respective to precisions. Method-2 would be more precise out of given two in case of standard deviation calculation.

\newpage
\section{An Identity}
\paragraph{}For identity task, Java code (Identity class) has considered given equation and calculated output for single precision and double precision data types. Output for same is as per below. \\

\begin{figure}[h]
	Output:\\
	Single Precision Input x = $2.5$  y = $3.5$ \\
	Single Precision Left Side: $2.5$  Right Side: $2.5$  Result: $true$ \\
	Double Precision Input x = $2.5848$  y = $3.5234$ \\
	Double Precision Left Side: $2.5848$  Right Side:  $2.584800000000005$  Result: $false$ \\ \\
	Single Precision Input x = $2.51$  y = $3.5$ \\
	Single Precision Left Side: $2.51$  Right Side: $2.5100002$  Result: $false$ 
	Double Precision Input x = $2.5848$  y = $3.5234$ \\
	Double Precision Left Side: $2.5848$  Right Side: $2.584800000000005$  Result: $false$
\end{figure}

\paragraph{} As per the above output for different inputs, Single precision till single decimal points satisfy the equation. if there are more than one decimals after points then even for single precision point equations fails even though it is mathematically correct. In case of Double precision it fails every time irrespective of given input.

\end{document}